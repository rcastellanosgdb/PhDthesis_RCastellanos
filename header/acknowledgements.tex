\begin{acknowledgements}
%
% \begin{small}
The work enclosed in this thesis has been partially supported by the Universidad Carlos III de Madrid through a PIPF
scholarship awarded on a competitive basis, and by the following research projects: ARTURO (Active contRol of Turbulence for sUstainable aiRcraft propulsiOn), ref. PID2019-109717RB-I00/AEI/10.13039/501100011033, funded by the Spanish State Research Agency (SRA); the 2020 Leonardo Grant for Researchers and Cultural Creators AEROMATIC (Active flow control of aerodynamic flows with machine learning), funded by the BBVA Foundation with grant number IN[20]\_ING\_ING\_0163; and GloWing Starting Grant, funded by the European Research Council (ERC), under grant agreement ERC-2018.StG-803082.\\

You are reading the feelings of a 27-year-old boy after four exhausting years of work, learning, pain, passion, desperation, self-development, anxiety, ambition, disillusionment, and, above all fun. It is due to these ups rather than to the downs that I want to express my gratitude to everyone supporting me during this long and ardours adventure. It is then a must to start with my advisors. Stefano, and Andrea, if I am right now writing these lines is because of you two. Stefano, you are the closest to an \textit{academic father} to me, you are my true mentor and reference to follow. I appreciate your support, not only in my academic work but in my daily life in general. Thanks for your patience, your unstoppable effort to make me improve and your willingness to listen to me anytime. Andrea, you are one of the most special human beings that I have ever met, in all possible senses of the word special. You have the capability of driving me crazy while teaching me at the same time. You are a tireless source of ideas, problems, fun and knowledge at the same time. Thanks for being like you are and, please, keep being so for the rest of your life. You may not realise, but you two are creating a \textit{school} with a bunch of henchmen after you. Talking about henchmen and goons, it is the turn of my friend and advisor in the shadows, Marco. You are the toughest but the most willing guy I know in this world. This rare but effective combination is why I want to thank you because you have been a real pillar and stable point during my doctorate path. Thanks for helping and teaching me in the lab, and thanks for letting me help you. If my experiments did not crash, it was because Marco fixed them while shouting and cursing.

Let me travel now to the Netherlands, to Delft in particular. This is what I did almost 4 years ago, without any idea about what I was going to do there but hoping to learn something to make the trip worthwhile. There I met Marios and his gang of PhD students who welcomed me as one of them for almost 6 months. Marios, thank you very much for your effort, patience and knowledge. You convinced me of staying on the dark side of academia when I was still unsure of myself. I am also tremendously thankful to Theo. I will always repeat in my mind \textit{lunchemos} and \textit{cafemos}. You are one of my closest friends, you are really \textit{samsing}. Thanks to Beto, Kaisheng, Hongxin and Gulia for making my stay in the lab a very pleasant time.

I would like to thank you, Bernd and Guy, for being so proactive and willing to help. You taught me and introduced me to the amazing world of machine learning control and you trusted me. Thank you for your support and the rich discussion. I really hope to keep working together in the future.

Estoy especialmente agradecido a mi gran amigo Carlos. Tú me acogiste y me enseñaste cuando era un traidor, y me has seguido enseñando y apoyando desde entonces, especialmente en los últimos meses de doctorado. Gracias por contar conmigo y por hacer planes a futuro en los que quieres que esté. Además, no me puedo olvidar de Gonzalo, con quien formarnos el equipo de las \textit{cibercerves}. Carlos, Gonzalo, no sabéis cuánto bien me hicisteis en uno de los peores momentos por los que he pasado. Una simple llamada me sacaba del pozo en el que estaba.

Es turno ahora de los currelas que me han sacado las castañas del fuego. Gracias a Gianfranco por ser `el ingeniero' que puso en marcha mi tesis. Gian, amigo, eres un tipo enorme, muy valioso en lo personal y lo profesional, gracias. A los Minions, Juancho e Issac. Chavales, vosotros sí que sois de lo que no hay. Gracias por no decir que \textit{no} a nada de lo que os propongo, por animarme cuando estaba rozando el suelo y por mantenerme espabilado cuando las cosas iban bien. Espero seguir disfrutando de vuestra compañía, ya sea en el trabajo o en el Abanades con un panini los jueves. Otro especial gracias a Miguel (y por extensión a su hermano). No me conocías de nada y me brindaste toda tu ayuda desde el primer momento. Simplemente gracias por estar dispuesto, por ser el técnico al otro lado del teléfono cuando te necesitábamos y gracias, sobre todo, por ser mi amigo.

Gracias a mis compañeros de despacho en la universidad. El \textit{triángulo de las bermudas} perdurará, gracias Juanma y Manu por ser mis mejores compañeros y amigos. Gracias a Caye, por alegrarnos las estancias en el despacho con tus cánticos espontáneos. Gracias a todos los compañeros de cervezas, Luca, Sensei Cini, Besu Griego, Iván (el Rubio). Pepa pig os agradece todos los grandes momentos.

Gracias a mi amiga María. Aunque nunca lograré enseñarte matemáticas, estoy orgulloso de lo que haces, de quien eres y te estoy muy agradecido por mantenerme vivo estos años. Gracias a mi amigo de vida, Varito. Aunque nos distanciemos, es increíble ver como seguimos teniendo siempre un rato para charlar desde los tres años.

Gracias a mis compañeros en el INTA. Gracias Alex y Javi por estar siempre dispuestos a echar una mano, por comprenderme y por animarme. En especial, gracias a Jaime, te has convertido en un gran amigo en pocos meses y me has hecho quererme un poco más. Jaime, eres un tío muy grande y vas a llegar muy lejos, espero que te acuerdes de mi cuando estés sentado a la mesa con Bezos y compañía. También quiero agradecer a Esther por su comprensión y por darme la oportunidad de trabajar con ella, formando un gran equipo y aprendiendo mucho en un campo nuevo para mi hasta ahora.

Gracias a Irene por haberme impulsado desde los 18 años. Ire, tú me has enseñado a ser como soy y tú me has motivado a estar donde estoy. Gracias por haber compartido conmigo 7 preciosos años, y gracias por seguir siendo mi gran amiga.

Es imposible nombrar a todas aquellas personas que me han apoyado y animado estos últimos años. Os pido disculpas si no os veis reflejados en estas líneas, pero sabed que estoy tremendamente agradecido.

Gracias a mi familia. A mis primos y tíos, en especial a mi tío Javi, el \textit{manager}, y a mi primo Giovanni. Gracias Giova por comprenderme y por escucharme, por animarme a hacer planes chulos y por esas cervecitas tan ricas a las que me vas a invitar por ser tu primo favorito. Gracias a mis abuelos. A Felipe, el yayo, a quien le habría encantado estar hoy aquí para verme siendo doctor. A mi yaya y a mi abuela, las mujeres más fuertes que he conocido. Y a mi abuelo, por estar siempre cerca, cuidándome y queriéndome por encima de todo.

A mi hermana. Pauli, a pesar de nuestros más y nuestros menos, debes saber que eres muy importante para mí. Gracias por ser la peor y la mejor de las hermanas. Gracias por haberme hecho crecer. Y, sobre todo, gracias por quererme como tú lo haces.

Acabo agradeciendo a las personas más importantes en mi vida, a mis padres. Mamá, papá, mirad lo que he conseguido. Esto es por vosotros. Gracias por ser, estar y dejarme ser desde que era un niño. Vuestro apoyo no tiene medida y nunca podré expresar mi gratitud.

% \end{small}
\end{acknowledgements}
