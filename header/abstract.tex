\begin{abstract}
	\noindent 
	The sustainable development of our society opens up concerns in several fields of engineering, including energy management, production, and the impact of our technology, being thermal management a common issue to be addressed. The investigation reported in this manuscript focuses on understanding, controlling and optimizing the physical processes involving convective heat transfer in turbulent wall-bounded flows. The content is divided into two main blocks, namely the investigation of classic open-loop active-control techniques to control heat transfer, and the technological development of machine-learning strategies to enhance the performance of flow control in the field of convective heat transfer. 
	
    The first block focuses on actuator technology, applying dielectric-barrier discharge (DBD) plasma actuators and a pulsed slot jet in crossflow (JICF), respectively, to control the convective heat transfer in a turbulent boundary layer (TBL) over a flat plate. In the former, an array of DBD plasma actuators is employed to induce pairs of counter-rotating, streamwise-aligned vortices embedded in the TBL to reduce heat transfer downstream of the actuation. The whole three-dimensional mean flow field downstream of the plasma actuator is reconstructed from stereoscopic particle image velocimetry (PIV). Infrared thermography (IR) measurements coupled with a heated thin foil provide ensemble-averaged convective heat transfer distributions downstream of the actuators. The combination of the flow field and heat transfer measurements provides a complete picture of the fluid-dynamic interaction of plasma-induced flow with local turbulent transport effects. The plasma-induced streamwise vortices are stationary and confined across the spanwise direction due to the action of the plasma discharge. The opposing plasma discharge causes a mass- and momentum-flux deficit within the boundary layer, leading to a low-velocity region that grows in the streamwise direction and which is characterised by an increase in displacement and momentum thicknesses. This low-velocity ribbon travels downstream, promoting streak-alike patterns of reduction in the convective heat transfer distribution. Near the wall, the plasma-induced jets divert the main flow due to the DBD-actuator momentum injection and the suction on the surrounding fluid by the emerging jets. The stationarity of the plasma-induced vortices makes them persistent far downstream, reducing the convective heat transfer.
    
    Conversely, the target of the second paper in this first block is to enhance convective heat transfer rather than reduce it. A fully modulated, pulsed, slot JICF is used to perturb the TBL. The slot-jet actuator, flush-mounted and aligned in the spanwise direction, is controlled based on two design parameters, namely the duty cycle ($DC$) and the pulsation frequency ($f$). Heat transfer and flow-field measurements are performed to characterise the control performance using IR thermography and planar PIV, respectively. A parametric study on $f$ and $DC$ is carried out to assess their effect on the heat transfer distribution. The vorticity fields are reconstructed from the Proper Orthogonal Decomposition (POD) modes, retrieving phase information. The flow topology is considerably altered by the jet pulsation, even compared to the case of a steady jet. The results show that both the jet penetration in the streamwise direction and the overall Nusselt number increase with increasing $DC$. However, the frequency at which the Nusselt number is maximised is independent of the duty cycle. A wall-attached jet rises from the slot accompanied by a pair of counter-rotating vortices that promote flow entrainment and mixing. Eventually, a simplified model is proposed which decouples the effect of $f$ and $DC$ in the overall heat transfer enhancement, with a good agreement with experimental data. The cost of actuation is also quantified in terms of the amount of injected fluid during the actuation, leading to conclude that the lowest duty cycle is the most efficient for heat transfer enhancement among the tested set.
	
    The second block of the thesis splits into a comparative assessment of machine learning (ML) methods for active feedback flow control and an application of linear genetic algorithms to an experimental convective heat transfer enhancement problem. First, the comparative study is carried out numerically based on a well-established benchmark problem, the drag reduction of a two-dimensional Kármán vortex street past a circular cylinder at a low Reynolds number ($Re=100$). The flow is manipulated with two blowing/suction actuators on the upper and lower side of a cylinder. The feedback employs several velocity sensors. Two probe configurations are evaluated: 5 and 11 velocity probes located at different points around the cylinder and in the wake. The control laws are optimized with Deep Reinforcement Learning (DRL) and Linear Genetic Programming Control (LGPC). Both methods successfully stabilize the vortex alley and effectively reduce drag while using small mass flow rates for the actuation. DRL features higher robustness with respect to variable initial conditions and noise contamination of the sensor data; on the other hand, LGPC can identify compact and interpretable control laws, which only use a subset of sensors, thus allowing reducing the system complexity with reasonably good results.
    
    The gained experience and knowledge of machine-learning methods motivated the last study enclosed in this thesis, which utilises linear genetic algorithm control (LGAC) to identify the best actuation parameters in an experimental application. The actuator is a set of six slot jets in crossflow aligned with the freestream. An open-loop optimal periodic forcing is defined by the carrier frequency ($f$), the duty cycle ($DC$) and the phase between actuators ($\phi$) as control parameters. The control laws are optimised with respect to the unperturbed TBL and the steady-jet actuation. The cost function includes wall convective heat transfer and the cost of the actuation, thus leading to a multi-objective optimisation problem. Surprisingly, the LGAC algorithm converges to the same frequency and duty cycle for all the actuators. This frequency is equivalent to the optimal frequency reported in the second study of the first block of this thesis. The performance of the controller is characterised by IR thermography and PIV measurements. The action of the jets considerably alters the flow topology compared to the steady-jet actuation,  yielding a slightly asymmetric flow field.  The phase difference between multiple jet actuation has shown to be very relevant and the main driver of flow asymmetry. A POD analysis concludes the shedding phenomena characterising the steady-jet actuation, while the optimised controller exhibits an elongated large-scale structure just downstream of the actuator.
    
    The investigation carried out in this thesis sheds some light on the application of different flow control strategies to the field of convective heat transfer. From the utilisation of plasma actuators and a single jet in cross flow to the development of sophisticated control logic, the results point to the exceptional potential of machine learning control in unravelling unexplored controllers within the actuation space. Ultimately, this work demonstrates the viability of employing sophisticated measurement techniques together with advanced algorithms in an experimental investigation, paving the way towards more complex applications involving feedback information.
		%
		\keywords{Convective heat transfer, flow control, machine learning, jet in cross flow, plasma actuator, turbulent boundary layer, reinforcement learning, genetic programming, genetic algorithms.}
		%
\end{abstract}