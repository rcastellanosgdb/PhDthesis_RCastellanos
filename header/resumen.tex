\begin{abstrakt}

	\noindent El desarrollo sostenible de nuestra sociedad abre preocupaciones en varios campos de la ingeniería, incluyendo la gestión de la energía, la producción y el impacto de nuestra tecnología, siendo la gestión térmica un tema común a tratar. La investigación que se presenta en este manuscrito se centra en la comprensión, el control y la optimización de los procesos físicos que implican la transferencia de calor por convección en flujos turbulentos de pared. El contenido se divide en dos bloques principales: la investigación de las técnicas clásicas de control activo de lazo abierto para controlar la transferencia de calor, y el desarrollo tecnológico de estrategias de aprendizaje automático para mejorar el rendimiento del control del flujo en el campo de la transferencia de calor por convección. 
    
    
    El primer bloque se centra en la tecnología de los actuadores, aplicando actuadores de plasma de descarga de barrera dieléctrica (dielectric barrier dicharge, DBD) y un chorro con forma de ranura pulsado en flujo cruzado (jet in crossflow, JICF), respectivamente, para controlar la transferencia de calor por convección en una capa límite turbulenta (turbulent boundary layer, TBL) sobre una placa plana. En el primero, se emplea un conjunto de actuadores de plasma DBD para inducir pares de vórtices contra-rotativos, alineados con la corriente e incrustados en la TBL para reducir la transferencia de calor aguas abajo de la actuación. El campo de flujo medio tridimensional completo aguas abajo del actuador de plasma se reconstruye a partir de la velocimetría de imagen de partículas estereoscópica (particle image velocimetry, PIV). Las mediciones de termografía infrarroja (IR) junto a una fina lámina calentada proporcionan distribuciones de transferencia de calor convectiva promediadas aguas abajo de los actuadores. La combinación de las mediciones del campo de flujo y de la transferencia de calor proporciona una imagen completa de la interacción fluido-dinámica del flujo inducido por el plasma con los efectos locales de transporte turbulento. Los vórtices en el sentido de la corriente inducidos por plasma son estacionarios y están confinados transversalmente debido a la acción de la descarga de plasma. La descarga de plasma en oposición causa un déficit de flujo de masa y de momento dentro de la capa límite, lo que conduce a una región de baja velocidad que crece en la dirección de la corriente y que se caracteriza por un aumento de los espesores de desplazamiento y de momento. Esta zona de baja velocidad se desplaza corriente abajo, promoviendo patrones de reducción similares a rayas en los que se reduce la transferencia de calor por convección. Cerca de la pared, los chorros inducidos por el plasma desvían el flujo principal debido a la inyección de momento del actuador DBD y a la succión sobre el fluido circundante por parte de los chorros emergentes. La estacionariedad de los vórtices inducidos por el plasma los hace persistentes aguas abajo, reduciendo la transferencia de calor por convección.
    
    Por el contrario, el objetivo del segundo trabajo de este primer bloque es mejorar la transferencia de calor por convección en lugar de reducirla. Se utiliza un JICF, pulsado y con forma de ranura, totalmente modulado para perturbar la TBL. El actuador de chorro, montado a ras y alineado en la dirección transversal, se controla en base a dos parámetros de diseño, a saber, el ciclo de trabajo ($DC$) y la frecuencia de pulsación ($f$). Se realizan mediciones de la transferencia de calor y del campo de flujo para caracterizar el rendimiento del control mediante termografía IR y PIV planar, respectivamente. Se lleva a cabo un estudio paramétrico de $f$ y $DC$ para evaluar su efecto en la distribución de la transferencia de calor. Los campos de vorticidad se reconstruyen a partir de los modos de descomposición ortogonal adecuada (POD), recuperando la información de fase. La topología del flujo se ve considerablemente alterada por la pulsación del chorro, incluso en comparación con el caso de un chorro estacionario. Los resultados muestran que tanto la penetración del chorro en la dirección de la corriente como el número Nusselt global aumentan con el incremento de $DC$. Sin embargo, la frecuencia a la que se maximiza el número Nusselt es independiente del ciclo de trabajo. Un chorro adherido a la pared sale de la ranura acompañado de un par de vórtices contrarrotantes que promueven el arrastre y la mezcla del flujo. Finalmente, se propone un modelo simplificado que desacopla el efecto de $f$ y $DC$ en la mejora global de la transferencia de calor, con un buen acuerdo con los datos experimentales. También se cuantifica el coste de la actuación en términos de la cantidad de fluido inyectado durante la actuación, llegando a la conclusión de que el ciclo de trabajo más bajo es el más eficiente para la mejora de la transferencia de calor entre el conjunto probado.

    El segundo bloque de la tesis se divide en una evaluación comparativa de los métodos de aprendizaje automático (machine learning, ML) para el control activo del flujo por retroalimentación y una aplicación de algoritmos genéticos lineales a un problema experimental de mejora de la transferencia de calor por convección. En primer lugar, el estudio comparativo se realiza numéricamente a partir de un problema de referencia bien establecido: la reducción de la resistencia aerodinámica de una calle de vórtices de Kármán bidimensional tras un cilindro circular a un número de Reynolds bajo ($Re=100$). El flujo se manipula con dos actuadores de soplado/succión en la parte superior e inferior de un cilindro. La retroalimentación emplea varios sensores de velocidad. Se evalúan dos configuraciones de sondas: 5 y 11 sondas de velocidad situadas en diferentes puntos alrededor del cilindro y en la estela. Las leyes de control se optimizan con el aprendizaje profundo por refuerzo (Deep Reinforcement Learning, DRL) y el control por programación genética lineal (Linear Genetic Programming Control, LGPC). Ambos métodos estabilizan con éxito la calle de vórtices y reducen de manera efectiva la resistencia al tiempo que usan caudales másicos pequeños para la actuación. El DRL se caracteriza por una mayor robustez con respecto a la variación de la condición inicial y a la contaminación por ruido de los datos de los sensores; por otro lado, el LGPC puede identificar leyes de control compactas e interpretables, que sólo utilizan un subconjunto de sensores, lo que permite reducir la complejidad del sistema con resultados razonablemente buenos.

    La experiencia adquirida y el conocimiento de los métodos de aprendizaje automático motivaron el último estudio incluido en esta tesis, que utiliza el control por algoritmo genético lineal (Linear Genetic Algorithm Control, LGAC) para identificar los mejores parámetros de actuación en una aplicación experimental. El actuador es un conjunto de seis chorros con forma de ranura en flujo cruzado y alineados con la corriente principal. Se define una ley de forzado periódica en lazo abierto mediante la frecuencia portadora ($f$), el ciclo de trabajo ($DC$) y la fase entre actuadores ($\phi$) como parámetros de control. Las leyes de control se optimizan con respecto a la TBL no perturbada y la actuación de chorro constante. La función de coste incluye la transferencia de calor por convección de la pared y el coste de la actuación, lo que da lugar a un problema de optimización multiobjetivo. Sorprendentemente, el algoritmo LGAC converge a la misma frecuencia y ciclo de trabajo para todos los actuadores. Esta frecuencia es equivalente a la frecuencia óptima reportada en el segundo estudio del primer bloque de esta tesis. El rendimiento del controlador se caracteriza mediante termografía IR y mediciones PIV. La acción de los chorros altera considerablemente la topología del flujo en comparación con la actuación de los chorros constantes, dando lugar a un campo de flujo ligeramente asimétrico. La diferencia de fase entre la actuación de múltiples chorros ha demostrado ser muy relevante y el principal impulsor de la asimetría del flujo. Un análisis POD concluye los fenómenos de desprendimiento de vórtices que caracterizan la actuación de chorro constante, mientras que el controlador optimizado muestra una estructura alargada a gran escala justo aguas abajo del actuador.
    
    La investigación llevada a cabo en esta tesis arroja algo de luz sobre la aplicación de diferentes estrategias de control de flujo en el campo de la transferencia de calor por convección. Desde la utilización de actuadores de plasma y un único chorro en flujo cruzado hasta el desarrollo de una sofisticada lógica de control, los resultados apuntan al excepcional potencial del control por aprendizaje automático para desentrañar controladores inexplorados dentro del espacio de actuación. En última instancia, este trabajo demuestra la viabilidad de emplear sofisticadas técnicas de medición junto con algoritmos avanzados en una investigación experimental, allanando el camino hacia aplicaciones más complejas que implican información de retroalimentación.

    %
    \palabrasclave{Transferencia de calor por convección, control de flujo, aprendizaje automático, chorro en flujo cruzado, actuador de plasma, capa límite turbulenta, aprendizaje por refuerzo, programación genética, algoritmos genéticos.}
    %
\end{abstrakt}